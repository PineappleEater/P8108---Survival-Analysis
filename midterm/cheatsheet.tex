\documentclass[10pt,letterpaper]{extarticle}
\usepackage[utf8]{inputenc}
\usepackage[T1]{fontenc}
\usepackage{multicol}
\usepackage{geometry}
\usepackage{amsmath,amsthm,amsfonts,amssymb}
\usepackage{enumitem}

\geometry{letterpaper,top=0.35cm,left=0.45cm,right=0.45cm,bottom=0.9cm}
\pagestyle{plain}
\setlength{\footskip}{0.5cm}

\makeatletter
\renewcommand{\section}{\@startsection{section}{1}{0mm}%
                                {-1.2ex plus -.4ex minus -.2ex}%
                                {0.5ex plus .2ex}%
                                {\normalfont\normalsize\bfseries}}
\makeatother

\setcounter{secnumdepth}{0}
\setlength{\parindent}{0pt}
\setlength{\parskip}{2.0pt plus 1.0ex}
\setlength{\abovedisplayskip}{3.0pt}
\setlength{\belowdisplayskip}{3.0pt}
\setlength{\itemsep}{1.0pt}
\setlength{\topsep}{2.0pt}

\newcommand{\h}[1]{\textbf{#1}}

\begin{document}
\raggedright
\footnotesize
\begin{multicols}{3}

\setlength{\columnsep}{4pt}
\setlength{\multicolsep}{4pt plus 1.5pt minus 1pt}

\begin{center}
     \Large{\textbf{Survival Analysis}} \\
     \normalsize{P8108 Midterm - Chapters 1-4} \\
     \small{Xuange Liang \\ xl3493@cumc.columbia.edu}
\end{center}

\section{Ch1: Basic Definitions \& Notation}

\h{Notation}: $h(t)=\lambda(t)$ (hazard), $H(t)=\Lambda(t)$ (cumulative hazard)

\h{Key relationships}: \\
$S(t) = 1-F(t) = e^{-H(t)}$; $F(t) = \int_0^t f(u)du$; $f(t) = -S'(t)$ \\
$h(t) = \frac{f(t)}{S(t)} = -\frac{d\log S(t)}{dt}$; $H(t) = \int_0^t h(u)du = -\log S(t)$ \\
$f(t) = h(t)S(t)$; $P(T>s+t|T>s) = \frac{S(s+t)}{S(s)}$

\h{Median}: $t_{0.5}$ where $S(t_{0.5})=0.5$; \h{Mean}: $E(T)=\int_0^\infty S(t)dt$

\h{Properties}: $S(0)=1$, $S(\infty)=0$, non-increasing, right-continuous. $h(t) \ge 0$ (can be $>1$)

\vspace{2pt}
\section{Ch1: Types of Censoring}

\h{Right censoring} (most common): \\
Event occurs after observation period ends \\
Observe $X = \min(T, C)$ where $C$=censoring time \\
$\delta = I(T \le C)$: event indicator (1=event, 0=censored)

\h{Left censoring}: event before study starts \\
\h{Interval censoring}: event in known interval

\h{Type I censoring}: fixed censoring time \\
\h{Type II censoring}: study ends after fixed \# events \\
\h{Random censoring}: censoring time random

\h{Left truncation}: only observe subjects surviving past entry $L$. Risk set: only those entered and not yet failed/censored. Differs from left censoring

\section{Ch1: Censoring Assumptions}

\h{Non-informative (independent) censoring}: \\
Censoring time independent of failure time \\
$P(T>t | T>u, \text{censored at } u) = P(T>t | T>u)$ \\
REQUIRED for valid KM estimator

\h{Informative censoring}: \\
Censoring related to failure mechanism \\
Violates standard methods - need special approaches

\h{Example of informative}: patients withdraw due to worsening condition

\h{Censoring distribution}: Reverse KM: treat censoring as "event". Check if censoring differs between groups. $\hat{G}(t) = P(C > t)$

\vspace{2pt}
\section{Ch2: Kaplan-Meier Estimator}

\h{Product-limit estimator}: \\
$$\hat{S}(t) = \prod_{t_i \le t} \left(1 - \frac{d_i}{r_i}\right) = \prod_{t_i \le t} \frac{r_i - d_i}{r_i}$$
where: \\
$t_1 < t_2 < \cdots$: ordered distinct event times \\
$d_i$: \# events at $t_i$ \\
$r_i$: \# at risk at $t_i$ (alive just before $t_i$)

\h{Risk set at $t_i$}: Include those who fail at $t_i$ and censored after $t_i$; exclude those censored at $t_i$ or failed before $t_i$

\h{Key}: $\hat{S}(t)$ step function, decreases only at event times

\h{Greenwood's variance formula}: \\
$$\text{Var}(\hat{S}(t)) = \hat{S}(t)^2 \sum_{t_i \le t} \frac{d_i}{r_i(r_i - d_i)}$$

\h{Standard error}: \\
$$SE(\hat{S}(t)) = \hat{S}(t) \sqrt{\sum_{t_i \le t} \frac{d_i}{r_i(r_i-d_i)}}$$

\h{KM-based cumulative hazard}: $\hat{\Lambda}_{KM}(t) = -\log \hat{S}_{KM}(t)$. Alternative to NA. Inverse: $\hat{S}_{KM}(t) = e^{-\hat{\Lambda}_{KM}(t)}$

\h{Variance of $\log\hat{S}(t)$}: \\
$$\text{Var}[\log\hat{S}(t)] = \sum_{t_i \le t} \frac{d_i}{r_i(r_i-d_i)}$$
Note: NO $\hat{S}^2$ factor! Used in log transformation CI

\h{CI for $\hat{S}(t)$ - Three methods}: \\
\textbf{1. Plain (linear)}: $\hat{S}(t) \pm 1.96 SE(\hat{S}(t))$ \\
Problem: may exceed $[0,1]$ \\
\textbf{2. Log transformation}: $SE[\log\hat{S}(t)] = \frac{SE(\hat{S})}{\hat{S}(t)}$ \\
CI: $\exp[\log\hat{S}(t) \pm 1.96 SE(\log\hat{S})]$ \\
\textbf{3. Log-log (preferred, default)}: $\theta = \log[-\log\hat{S}(t)]$ \\
$SE(\theta) = \frac{\sqrt{\sum d_i/(r_i(r_i-d_i))}}{|\log\hat{S}(t)|}$ \\
CI: $\exp\{-\exp(\theta \pm 1.96 SE(\theta))\}$ \\
Always stays in $[0,1]$, best for extreme $\hat{S}(t)$ values

\h{Percentiles}: $p$-th: smallest $t_p$ where $\hat{S}(t_p) \le 1-p$. Median: $\hat{S}(t_{0.50}) = 0.50$. If $\hat{S}(t)$ never reaches threshold, undefined

\h{Backward calculation}: Given $\hat{S}(t)$ product form \& $n$, find $d_i$, $r_i$, $c_i$:

\h{Step-by-step}: 1) From $\hat{S}(t_i)/\hat{S}(t_{i-1}) = (r_i-d_i)/r_i$, read denominator $\Rightarrow r_i$, then $d_i=r_i-\text{numerator}$. 2) Compute $c_i$: if $r_1<n$, then $n-r_1$ censored before $\tau_1$; otherwise $c_i = (r_i-d_i) - r_{i+1}$. 3) Verify: $\sum d_i + \sum c_i = n$ (critical!)

\h{Ex (Sample Q1)}: $n=13$, $\hat{S}(\tau_1)=10/11 \Rightarrow r_1=11, d_1=1, c_{\text{early}}=13-11=2$. $\hat{S}(\tau_2)=(10/11)(9/10) \Rightarrow r_2=10, d_2=1, c_1=(11-1)-10=0$. Continue. Final check: $\sum d_i=6, \sum c_i=7, 6+7=13$ $\checkmark$

\h{Trap}: If product starts $<1$, early censoring! $r_1 = $ first denominator, NOT $n$

\h{Backward calc}: $n=10$, $\hat{S}(\tau_1)=\frac{8}{9}$, $\hat{S}(\tau_2)=\frac{8}{9} \times \frac{6}{7}$, $\hat{S}(\tau_3)=\frac{8}{9} \times \frac{6}{7} \times \frac{4}{5}$ $\Rightarrow$ $r_1=9, d_1=1, c_0=1$; $r_2=7, d_2=1, c_1=1$; $r_3=5, d_3=1, c_2=1$; $c_3=4$. Check: $3+7=10$ $\checkmark$

\h{RMST}: $\text{RMST}(\tau) = \int_0^\tau \hat{S}(t)dt$ (area under KM to $\tau$). Use when median not reached

\h{KM example}: Data: 2, 3+, 5, 6. $t=2$: $r=4, d=1, \hat{S}(2)=\frac{3}{4}=0.75$. $t=3$: censored, $\hat{S}=0.75$. $t=5$: $r=2, d=1, \hat{S}(5)=0.75 \times \frac{1}{2}=0.375$. Greenwood: $\text{Var}[\hat{S}(5)]=(0.375)^2[\frac{1}{12}+\frac{1}{2}]=0.082$. Check: $\sum d_i + \sum c_i = n$

\vspace{2pt}
\section{Ch2: Other Estimators}

\h{Nelson-Aalen (NA) cumulative hazard}: \\
$$\hat{H}_{NA}(t) = \sum_{t_i \le t} \frac{d_i}{r_i}$$
Variance: $\text{Var}[\hat{H}_{NA}(t)] = \sum_{t_i \le t} \frac{d_i}{r_i^2}$

\h{Fleming-Harrington estimator}: \\
$$\hat{S}_{FH}(t) = \exp(-\hat{H}_{NA}(t))$$
Close to $\hat{S}_{KM}$ when hazards small. Uses NA for cumulative hazard

\h{Comparison}: NA: direct $H(t)$ estimate; KM: direct $S(t)$ estimate. $\hat{\Lambda}_{KM}(t) \approx \hat{H}_{NA}(t)$ for small hazards

\h{Life table (actuarial)}: For grouped data in $[t_j, t_{j+1})$: effective risk set $r_j^* = r_j - c_j/2$, then $\hat{q}_j = d_j/r_j^*$, $\hat{S}(t_{j+1}) = \hat{S}(t_j)(1-\hat{q}_j)$

\h{NA vs KM relationship}: \\
KM: $\hat{S}(t) = \prod (1-\frac{d_i}{r_i})$ \\
NA: $\hat{H}_{NA}(t) = \sum \frac{d_i}{r_i}$, $\hat{S}_{FH}(t) = e^{-\hat{H}_{NA}(t)}$ \\
When $d_i/r_i$ small: $1-\frac{d_i}{r_i} \approx e^{-d_i/r_i}$ \\
$\Rightarrow \hat{S}_{KM}(t) \approx \hat{S}_{FH}(t)$ and $-\log\hat{S}_{KM}(t) \approx \hat{H}_{NA}(t)$

\vspace{2pt}
\section{Ch3: Comparing Two Groups}

\h{Logrank test (Mantel-Haenszel)}: \\
$H_0: S_1(t) = S_2(t)$ for all $t$

At each event time $t_i$:
\begin{align*}
E_{1i} &= \frac{r_{1i} \cdot d_i}{r_i} \quad \text{(expected in group 1)} \\
V_i &= \frac{r_{1i} r_{2i} d_i (r_i - d_i)}{r_i^2(r_i-1)} \quad \text{(variance)}
\end{align*}
where $r_i = r_{1i} + r_{2i}$, $d_i = d_{1i} + d_{2i}$

Test statistic:
$$\chi^2_{LR} = \frac{[\sum_i (O_{1i} - E_{1i})]^2}{\sum_i V_i} \sim \chi^2_1$$

\h{Weights}: Logrank=equal; Wilcoxon: $\chi^2_W = [\sum r_i(O_{1i}-E_{1i})]^2/\sum r_i^2 V_i$ (early); $G^{\rho,\gamma}$: $w_i = \hat{S}^{\rho}(1-\hat{S})^\gamma$, $(\rho,\gamma)$: $(0,0)$=logrank, $(1,0)$=early, $(0,1)$=late

\h{Test selection}: Logrank when PH holds. Curves cross: Wilcoxon or weighted tests. With confounders: stratified logrank

\h{Stratified logrank}: \\
For $K$ strata, sum over strata:
$$\chi^2 = \frac{[\sum_{k=1}^K \sum_i (O_{1ik} - E_{1ik})]^2}{\sum_{k=1}^K \sum_i V_{ik}}$$

\h{Logrank steps}: 1) Pool all unique event times. 2) At each $t_i$: count $r_{1i}, r_{2i}, d_{1i}, d_{2i}$. 3) Calculate $E_{1i}, V_i$. 4) Sum: $O_1=\sum d_{1i}$, $E_1=\sum E_{1i}$, $V=\sum V_i$. 5) $\chi^2=(O_1-E_1)^2/V \sim \chi^2_1$

\h{Key points}: Pool times first. Skip censoring times. $r_i$ from pooled data. Sum over ALL event times

\vspace{2pt}
\section{Ch4: Cox Proportional Hazards}

\h{Model form}: \\
$$h(t, Z) = h_0(t) \exp(\beta_1 Z_1 + \cdots + \beta_p Z_p)$$
$$h(t, Z) = h_0(t) e^{\beta^T Z}$$

\h{Components}: \\
$h_0(t)$: baseline hazard (when all $Z=0$) \\
$Z = (Z_1, \ldots, Z_p)^T$: covariate vector \\
$\beta = (\beta_1, \ldots, \beta_p)^T$: coefficients

\h{Semi-parametric}: \\
No parametric form assumed for $h_0(t)$ \\
Estimate $\beta$ without specifying $h_0(t)$

\h{Proportional hazards property}: \\
$$\frac{h(t, Z^{(1)})}{h(t, Z^{(0)})} = \frac{h_0(t)e^{\beta^T Z^{(1)}}}{h_0(t)e^{\beta^T Z^{(0)}}} = e^{\beta^T(Z^{(1)}-Z^{(0)})}$$
Ratio independent of $t$ (constant over time)

\h{Hazard ratio (HR)}: \\
$$HR = e^{\beta^T(Z^{(1)} - Z^{(0)})}$$
For binary $Z$: $HR = e^\beta$ \\
For continuous $Z$ ($c$-unit change): $HR = e^{\beta c}$

\h{Interpretation}: \\
$\beta > 0$: $HR > 1$, increased hazard (worse survival) \\
$\beta < 0$: $HR < 1$, decreased hazard (better survival) \\
$\beta = 0$: $HR = 1$, no effect \\
\h{HR meaning}: HR=2 means hazard doubles; HR=0.5 means hazard halves. NOT survival probability!

\h{Log-linear model}: \\
$\log(HR) = \beta^T Z$ \\
Linear in log-hazard scale

\h{Partial likelihood}: \\
Cox (1972, 1975) developed partial likelihood \\
Conditions on risk sets at each failure time \\
Avoids specifying $h_0(t)$ \\
$$L(\beta) = \prod_{i=1}^D \frac{\exp(\beta^T Z_{(i)})}{\sum_{j \in R(t_{(i)})} \exp(\beta^T Z_j)}$$
where $R(t_i)$ = risk set at time $t_i$

\h{Estimation}: \\
Maximize $\log L(\beta)$ using Newton-Raphson \\
Obtain $\hat{\beta}$ and $\text{Var}(\hat{\beta})$ from information matrix

\section{Ch4: Inference for Cox Model}

\h{Three tests for $H_0:\beta=0$}:
1. \h{Wald}: $\chi^2_W = \frac{\hat{\beta}^2}{Var(\hat{\beta})} \sim \chi^2_1$
2. \h{Score}: $\chi^2_S = U(0)^T I(0)^{-1} U(0)$ (logrank for single binary covariate)
3. \h{Partial LR}: $\chi^2_{LR} = 2[\log L(\hat{\beta}) - \log L(\beta_0)]$ (most reliable)

Generally: $\chi^2_W \le \chi^2_{LR} \le \chi^2_S$ for small effects

\h{CI for HR}: $95\%$ CI = $e^{\hat{\beta} \pm 1.96 SE(\hat{\beta})}$. CI excludes 1 $\Leftrightarrow$ $p<0.05$

\h{Overall test}: For $H_0: \beta_1 = \cdots = \beta_p = 0$, use partial LR with $p$ df

\h{Confounding}: $>10\%$ change in $\hat{\beta}$ suggests confounding

\h{Nested model comparison}: \\
For nested models (Model 0 $\subset$ Model 1): \\
$H_0$: extra parameters = 0 \\
$\chi^2_{LR} = 2[\log L_1(\hat{\beta}_1) - \log L_0(\hat{\beta}_0)]$ \\
df = \# extra parameters in Model 1

\h{Individual coefficient test}: \\
Wald: $Z = \frac{\hat{\beta}_j}{SE(\hat{\beta}_j)} \sim N(0,1)$ or $\chi^2 = Z^2 \sim \chi^2_1$ \\
95\% CI: $\hat{\beta}_j \pm 1.96 \times SE(\hat{\beta}_j)$ \\
For HR: $e^{\hat{\beta}_j \pm 1.96 \times SE(\hat{\beta}_j)}$

\section{Ch4: Tied Times \& Other Topics}

\h{Handling ties}: Breslow (fast), Efron (better, default in R), Exact (slow). Use Efron for most cases

\h{Breslow method}: At time $t_i$ with $d_i$ tied deaths: \\
$$L_i(\beta) = \frac{\prod_{j \in D_i} \exp(\beta^T Z_j)}{\left[\sum_{k \in R_i} \exp(\beta^T Z_k)\right]^{d_i}}$$
Approximation: uses same risk set for all tied deaths

\h{Efron method}: Better approximation for ties \\
Let $S_R = \sum_{k \in R_i} e^{\beta^T Z_k}$, $S_D = \sum_{j \in D_i} e^{\beta^T Z_j}$
$$L_i(\beta) = \frac{\prod_{j \in D_i} e^{\beta^T Z_j}}{\prod_{l=1}^{d_i} \left[S_R - \frac{l-1}{d_i}S_D\right]}$$
Adjusts risk set by removing fraction of tied deaths

\h{Practical notes}: Breslow/Efron give similar results when few ties. Efron preferred when $>$5\% ties

\section{Ch4: Stratified Cox Model}

When PH assumption violated for a covariate:

\h{Stratified model}: \\
$$h_k(t, Z) = h_{0k}(t) e^{\beta^T Z}$$
Different baseline $h_{0k}(t)$ for stratum $k$ \\
Common $\beta$ across strata

\h{Stratification vs adjustment}: \\
Stratify: don't estimate effect, allow different baselines \\
Adjust: estimate effect, assume common baseline

\h{Interaction}: $h(t|Z_1,Z_2) = h_0(t)e^{\beta_1 Z_1+\beta_2 Z_2+\beta_3 Z_1 Z_2}$. Effect of $Z_1$ at $Z_2=c$: $HR=e^{\beta_1+\beta_3 c}$. Test: $H_0:\beta_3=0$

\h{Multiple covariates}: $HR = \exp[\beta_1(Z_1^{(1)}-Z_1^{(0)})+\beta_2(Z_2^{(1)}-Z_2^{(0)})]$

\h{Categorical (>2 levels)}: Use dummy variables. 3-level: 2 indicators. Overall test: LR with 2 df

\h{Variance-Covariance matrix}: \\
For $\hat{\beta} = (\hat{\beta}_1, \hat{\beta}_2)^T$: \\
$$\text{Var}(\hat{\beta}_1 + \hat{\beta}_2) = \text{Var}(\hat{\beta}_1) + \text{Var}(\hat{\beta}_2) + 2\text{Cov}(\hat{\beta}_1, \hat{\beta}_2)$$
Important for linear combinations of parameters (e.g., interaction effects)

\h{Baseline estimation}: $\hat{h}_0(t_i) = \frac{d_i}{\sum_{j \in R_i} e^{\hat{\beta}^T Z_j}}$; $\hat{H}_0(t)=\sum \hat{h}_0(t_i)$; $\hat{S}_0(t)=e^{-\hat{H}_0(t)}$

\h{Predicted survival}: $\hat{S}(t|Z) = [\hat{S}_0(t)]^{\exp(\hat{\beta}^T Z)}$. Precision: \# events, not subjects

\section{Ch4: Examples \& Interpretation}

\h{Basic HR}: $h(t)=h_0(t)e^{0.5 X_1+0.03 X_2}$ ($X_1$=trt, $X_2$=age). Trt vs control: $HR=e^{0.5}=1.65$; 10y age $\uparrow$: $HR=e^{0.3}=1.35$; Trt 60yo vs control 50yo: $HR=e^{0.5+0.3}=e^{0.8}=2.23$

\h{CI to SE}: HR=1.5 (1.2, 1.9) $\Rightarrow$ $\hat{\beta}=\log(1.5)=0.405$; $SE(\hat{\beta})=[\log(1.9)-\log(1.2)]/[2(1.96)]=0.117$; $Z=0.405/0.117=3.46$, $p<0.001$

\h{Stratified}: $h(t)=h_{0,stage}(t)e^{-0.971 X_E+0.003 X_A}$. ER+ vs ER- (same age/stage): $HR=e^{-0.971}=0.38$. Different stages: CANNOT compare!

\h{Interaction}: $h(t)=h_0(t)e^{\beta_1 Z_1+\beta_2 Z_2+\beta_3 Z_1 Z_2}$. Effect of $Z_1$ at $Z_2=0$: $HR=e^{\beta_1}$; at $Z_2=1$: $HR=e^{\beta_1+\beta_3}$

\h{Reading KM plots}: Step function; median = first $t$ where $\hat{S}(t) \le 0.5$; crossing curves $\Rightarrow$ PH violated; use formal tests (logrank/Cox); CI overlap is not a significance test; \\\# at risk shown in table below

\vspace{2pt}
\section{Ch4: Checking PH Assumption}

\h{PH assumption}: $h_1(t)/h_0(t) = HR$ (constant) $\Rightarrow S_1(t) = [S_0(t)]^{HR}$ (multiplicative, NOT additive)

\h{On KM plot} ($S(t)$ vs $t$): Curves NOT parallel! Distance changes over time. Non-crossing $\Rightarrow$ PH plausible. Crossing $\Rightarrow$ PH violated

\h{On log-log plot} ($\log[-\log S(t)]$ vs $t$): Curves ARE parallel if PH holds. Vertical distance = $\log(HR)$. Why: $\log[-\log S_1(t)] = \log[-\log S_0(t)] + \log(HR)$

\h{Tests}: 1) KM non-crossing. 2) Log-log parallel. 3) \texttt{cox.zph()} in R

\h{If violated}: Stratify (different $h_0$, don't estimate effect)

\vspace{2pt}
\section{Key Formulas \& Values}

\h{Critical values}: $z_{0.975}=1.96$, $z_{0.995}=2.576$, $\chi^2_{1,0.95}=3.84$, $\chi^2_{2,0.95}=5.99$

\h{Useful logs/exp}: $\log(2)=0.693$, $\log(0.5)=-0.693$, $e^{0.5}\approx 1.65$, $e^{-0.5}\approx 0.61$, $e^1\approx 2.72$, $e^{-1}\approx 0.37$

\h{More useful values}: $\log(3)=1.099$, $\log(0.38)\approx -0.97$, $e^{0.3}\approx 1.35$, $e^{-0.97}\approx 0.38$

\h{Conversions}: $\beta=\log(HR)$; $HR=e^\beta$; $\hat{H}(t)=-\log\hat{S}(t)$; $\hat{S}(t)=e^{-\hat{H}(t)}$

\h{Percentage change in hazard}: \\
HR = 1.5 means 50\% increase: $(HR-1) \times 100\%$ \\
HR = 0.6 means 40\% decrease: $(1-HR) \times 100\%$

\h{Cox prediction}: Individual survival $\hat{S}(t|Z)=[\hat{S}_0(t)]^{\exp(\hat{\beta}^T Z)}$. For $c$-unit change: $HR=e^{\beta c}$. Multiple changes: $HR=\exp[\beta_1\Delta Z_1+\beta_2\Delta Z_2]$

\h{Variance formulas}: Greenwood uses $\hat{S}^2$. For $\log\hat{S}$: $\text{Var}[\log\hat{S}]=\sum_{t_i \le t}\frac{d_i}{r_i(r_i-d_i)}$ (no $\hat{S}^2$ factor!)

\h{Partial likelihood (KEY CONCEPT)}: At each failure time $t_i$, conditional prob that individual $i$ fails given one failure in risk set $R_i$:
$$L_i(\beta) = \frac{\exp(\beta^T Z_i)}{\sum_{j \in R_i} \exp(\beta^T Z_j)}$$
Full partial likelihood: $L(\beta) = \prod_{i=1}^D L_i(\beta)$ over all $D$ events. Maximize $\log L(\beta)$ to get $\hat{\beta}$

\h{Log partial likelihood}: \\
$$\log L(\beta) = \sum_{i=1}^D \left[\beta^T Z_i - \log\left(\sum_{j \in R_i} \exp(\beta^T Z_j)\right)\right]$$
Score: $U(\beta) = \frac{\partial \log L}{\partial \beta}$; Information: $I(\beta) = -\frac{\partial^2 \log L}{\partial \beta \partial \beta^T}$

\h{Baseline hazard (Breslow estimator)}: After getting $\hat{\beta}$:
$$\hat{h}_0(t_i) = \frac{d_i}{\sum_{j \in R_i} \exp(\hat{\beta}^T Z_j)}$$
Then $\hat{H}_0(t) = \sum_{t_i \le t} \hat{h}_0(t_i)$, $\hat{S}_0(t) = \exp(-\hat{H}_0(t))$

\h{Relationship to KM}: When $\beta=0$ (no covariates), Cox model reduces to KM estimator

\h{Exponential}: $h(t)=\lambda$ (constant), $S(t)=e^{-\lambda t}$, $H(t)=\lambda t$. Median$=0.693/\lambda$. Mean$=1/\lambda$. Memoryless: $P(T>s+t|T>s)=P(T>t)$. MLE: $\hat{\lambda}=d/\sum_{i=1}^n X_i$

\h{Weibull}: $h(t)=\alpha\lambda t^{\alpha-1}$, $S(t)=e^{-\lambda t^\alpha}$. $\alpha>1$: $\uparrow$ hazard; $\alpha<1$: $\downarrow$ hazard; $\alpha=1$: exponential. Check: plot $\log H(t)$ vs $\log t$ linear

\h{Log-normal}: Non-monotone hazard (increases then decreases)

\vspace{2pt}
\section{Interpretation Templates}

\h{Compare treatments}: \\
\textbf{No diff}: "No difference between [A] vs [B] in [outcome]. KM curves [nearly identical], HR=[val] (95\% CI: [...]), $p=[val]$ (not sig)." \\
\textbf{Sig diff}: "[Advantage/Disadvantage] to [A] vs [B] in [outcome]. KM curves [separate], HR=[val] (95\% CI: [...]), $p=[val]$ (sig)." \\
Note: HR$<$1 favors exposed (lower hazard); HR$>$1 favors unexposed

\h{PH check}: \\
\textbf{OK}: "KM curves [parallel/non-crossing], PH reasonable. [Further diagnostics possible.]" \\
\textbf{Violated}: "KM curves [cross/converge], PH may be violated. [Consider stratified/time-dep.]"

\h{Cox model format}: \\
Binary $W$: $\lambda(t,W)=\lambda_0(t)e^{\beta W}$, $\beta=\log(HR)$ \\
Continuous $X$: HR for $c$-unit $\uparrow$: $e^{\beta c}$ \\
Multiple: $\lambda(t,Z)=\lambda_0(t)e^{\beta_1 Z_1+\beta_2 Z_2+\cdots}$ \\
Interaction: Effect of $Z_1$ at $Z_2=0$: $HR=e^{\beta_1}$; at $Z_2=1$: $HR=e^{\beta_1+\beta_3}$

\h{HR interpretation}: HR$>$1: $[(HR-1)\times 100]\%$ $\uparrow$ hazard; HR$<$1: $[(1-HR)\times 100]\%$ $\downarrow$ hazard

\vspace{2pt}
\section{Quick Reference}

\h{Test equivalences}: Score test for single binary covariate = Logrank test; Logrank = $G^{0,0}$ (equal weights); Wilcoxon = $G^{1,0}$ (early differences)

\h{Key inequalities}: $\chi^2_W \le \chi^2_{LR} \le \chi^2_S$; For small $x$: $1-x \approx e^{-x}$, $\log(1+x) \approx x$; Delta method: $\text{Var}[g(X)] \approx [g'(\mu)]^2 \text{Var}(X)$

\h{Assumptions}: KM: non-informative censoring; Logrank: PH + non-informative censoring; Cox PH: proportional hazards + non-informative censoring

\end{multicols}
\end{document}

